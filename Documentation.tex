\documentclass[12pt, letterpaper, twoside]{article}
\usepackage[utf8]{inputenc}
\usepackage{hyperref}%it helps when we write website so we can open it when we click on it. 
\usepackage{comment}

\title{Mobile app for fake news detection game}
\author{Karam Tamim Yanis and Minh Christian Tran}
\date{26 january 2021}



\begin{document}

\begin{titlepage}
\maketitle
\thispagestyle{empty}%delete page number for the contents
\end{titlepage}
\pagenumbering{roman}

\section*{abstract}
\addcontentsline{toc}{section}{\numberline{}abstract}% this code let the content put the page number for the abstract
This is a simple paragraph at the beginning of the 
document. A brief introduction about the main subject.
\cleardoublepage


% start page content 
\tableofcontents
\thispagestyle{empty}%delete page number for the contents
\cleardoublepage%this make 2 page empty after the tableofcontents
% end page content 

\setcounter{page}{1}%start numbering from chapter1 or intruduction
\pagenumbering{arabic}

\section{Introduction}
\subsection{Motivation}
% https://en.wikipedia.org/wiki/Fake_news
In the information era we live in today, where information is crucial among people, being fed misinformation can cause dangerous situations in our society. As new information and stories keeps being published at a faster rate without verification, figuring out what is true and what is false becomes increasingly harder to do. Fake news has gone from being sent via emails to practically being everywhere on the Internet, especially in social medias. In the 21th century, the ability to create fake news has never been easier than before thanks to the popularity of social media. People are able to generate financial profit through social media by deceiving readers into clicking links, thus maximizing traffic and profit. Creating fake news has even become a job for some as generating profit and spreading those fake news has become so easy. \\ % 21st century in the wiki link

% https://www.dictionary.com/browse/fake-news
According to Dictionary.com, the definition to fake news is: "false news stories, often of a sensational nature, created to be widely shared or distributed for the purpose of generating revenue, or promoting or discrediting a public figure, political movement, company, etc". Social medias such as Facebook, twitter, Instagram and blogs have recently become popular news providers, and along with the rise of news on social media, fake news become more prevalent. Teens, especially, are vulnerable to these fake news as they tend to use social media to read news and observe what is happening in the world.\\ 

Imagine here a simple scenario, such as you are in the midst of a theater full of audience. Suddenly everyone around you begins to panic, looking for a way out. So what will you do? Why? Your senses will tell you that others are moving frantically. But the social explanation that you crystallize for this information is what will tell you the most important thing you have to know, which is that these people think that something bad is happening, and that you might try to escape as well.\\

The above is at least one possible explanation for the panic you see around you, in this supposed scenario. Maybe you - or they - are wrong. It may be related to a false alarm, or to a portion of the play that was misunderstood. Therefore, accurately "reading" social information is a basic skill, and most of us devote an enormous amount of effort to mastering and practicing it.\\
Fake news has become frighteningly leaked in many countries of  world to spread an atmosphere of panic and fear of disease or to mix facts about the Corona pandemic and Corona vaccine, as the period of crisis is the best environment for the emergence of false news.\\
The spread of false news increased after the spread of social networking sites such as Facebook and Twitter, especially with the large number of news pages on Facebook, which may publish false news in order to increase the number of followers and profit.\\
 News deserves to be verified, then it is necessary to investigate the source who provided this content, is it known with a clear identity, is it among the citizen journalists or well-known activists? Has he ever provided reliable news? Can the information received be verified from various other sources?.\\
\subsection{Goal}
There are no effective solutions to combat misinformation, fact-checking technology can help solve this problem as long as it remains free of political and economic influences.\\
The news should also be taken from a reliable source, as it supports the news with official documents and photos.\\
In this regard, a number of websites have produced means that help to verify the authenticity of the news such as \url{https://www.politifact.com/} , in addition to searching for other reliable and popular sites that publish the same news, as well as verifying the pictures and videos that are relating.\\
Educating individuals and society about the danger of fake news remains one of the effective ways to combat this type of content and publications.\\
The main goal of this thesis is to shed light on the importance of uncovering false news and educate people in a fun and exiting way on the need to verify the truth of news they read on social media or through traditional media.\\
\subsection{Related Work}
\subsection{Outline}

\section{Theoretical Background}
\section{Bibliography}




\end{document}

